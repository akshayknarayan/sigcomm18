\section{Related Work}
\label{s:relwork}
%The CCP aims to provide a API for congestion control that decouples congestion control logic from the datapath, while keeping accurate measurement information from the datapath.
\paragrapha{Congestion Control API}
Linux includes a pluggable TCP API~\cite{lwn-pluggable-tcp}, which exposes various statistics for every connection, including delay, rates averaged over the past RTT, ECN information, timeouts, and packet loss.
icTCP~\cite{icTCP} is a modified TCP stack in the Linux kernel that allows user-space programs to modify specific TCP-related variables, such as the congestion window, slow start threshold, receive window size, and retransmission timeout. QUIC~\cite{quic} also offers pluggable congestion control. CCP extracts common primitives for congestion control across multiple datapaths, supporting off-datapath programmability with the performance of in-datapath implementations. 

\paragrapha{Kernel fastpath execution}
eBPF~\cite{ebpf} allows developers to run code safely inside the Linux Kernel by attaching probes to kernel functions; these can be used for debugging purposes. TCP BPF~\cite{tcpbpf} is an extension on top of BPF that allows customization of TCP connection settings, such as the TCP buffer size or SYN RTO, based on socket information such as IP address and port number. While it may be possible to develop a BPF backend to execute CCP fastpath programs, unlike \ct{libccp}, this method is not portable to other datapaths. Exploring BPF support in CCP remains future work.
